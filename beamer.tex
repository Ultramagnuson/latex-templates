%\documentclass[t,xcolor=dvipsnames]{beamer}
\documentclass[t,xcolor=dvipsnames,handout]{beamer}
\usepackage{graphics,multimedia,tikz,amssymb,graphicx,mathrsfs,color}
\usepackage{xcolor,colortbl}% http://ctan.org/pkg/xcolor
\usepackage{amsmath,amsthm,amsfonts,latexsym,amssymb,amscd,xcolor,leqno}
\usepackage{mathrsfs}
\usepackage{tikz}
\usepackage[english]{babel}
\usepackage[latin1]{inputenc}
\usepackage{times}
\usepackage[T1]{fontenc}
\usetikzlibrary{decorations.pathreplacing}
\usepackage{circuitikz}
\usetikzlibrary{arrows,shapes.gates.logic.US,shapes.gates.logic.IEC,calc}
\tikzstyle{branch}=[fill,shape=circle,minimum size=3pt,inner sep=0pt]

\setbeamertemplate{navigation symbols}{}

\theoremstyle{theoremFermat}
\newtheorem{theoremFermat}[theorem]{Theorem (Fermat)}

\theoremstyle{reimann}
\newtheorem{reimann}[theorem]{Reimann Hypothesis}

\theoremstyle{ACExample}
\newtheorem{ACExample}[theorem]{Almost Counterexample}

\mode<presentation>
{


\useoutertheme{infolines} 
  \usetheme{Warsaw}
  %\usetheme{Berkeley}
%%%%%%  \usetheme{Madrid}
  %\usetheme{default}
 %\usetheme{Boadilla}
%  \usetheme{Pittsburgh}
%  \usetheme{Rochester} %[ works best as \usetheme[height=7mm]{Rochester} ]
%  \usetheme{Copenhagen}
%  \usetheme{Singapore}
%  \usetheme{Malmoe}
%  \usetheme{Szeged}

%\usecolortheme[named=PineGreen]{structure} 
%\usecolortheme[named=Apricot]{structure} 
%\usecolortheme[named=Gray]{structure} 
%\usecolortheme[named=Tan]{structure} 
%\usecolortheme[named=ForestGreen]{structure} 
%\usecolortheme[named=Green]{structure} 
%\usecolortheme[named=SeaGreen]{structure} 
%\usecolortheme[named=LimeGreen]{structure} 
%\usecolortheme[named=OliveGreen]{structure} 
%\usecolortheme[named=SpringGreen]{structure} 
%\usecolortheme[named=JungleGreen]{structure} 
%\usecolortheme[named=Aquamarine]{structure} 
%\usecolortheme[named=Emerald]{structure} 
%\usecolortheme[named=BlueGreen]{structure} 
%\usecolortheme[named=TealBlue]{structure} 
\usecolortheme[named=NavyBlue]{structure} 
%\usecolortheme[named=DarkOrchid]{structure} 
%\usecolortheme[named=Periwinkle]{structure} 
%% more colors in /usr/share/texmf/tex/latex/graphics/dvipsnam.def 

\usefonttheme{serif}

}


\def\wec#1{{\mathbf #1}}


\title{Saturated models}

%%%\subtitle
%%%{Presentation Subtitle} % (optional)

%%%\author[Author, Another] % (optional, use only with lots of authors)
%%%{F.~Author\inst{1} \and S.~Another\inst{2}}
%%%% - Use the \inst{?} command only if the authors have different
%%%%   affiliation.
%\author{Kearnes, Kiss, Szendrei}


%%%\institute[Universities of Somewhere and Elsewhere] % (optional, but mostly needed)
%%%{
%%%  \inst{1}%
%%%  Department of Computer Science\\
%%%  University of Somewhere
%%%  \and
%%%  \inst{2}%
%%%  Department of Theoretical Philosophy\\
%%%  University of Elsewhere}
%%%% - Use the \inst command only if there are several affiliations.
%%%% - Keep it simple, no one is interested in your street address.

%%%\date[Short Occasion] % (optional)
%%%{Date / Occasion}
%\date{January 20, 2014\\ University of Waterloo}
\vskip -.2in

\date[]{}


%%%% If you have a file called "university-logo-filename.xxx", where xxx
%%%% is a graphic format that can be processed by latex or pdflatex,
%%%% resp., then you can add a logo as follows:

%%\pgfdeclareimage[height=2cm]{godel}{godel.jpg}
%%\logo{\pgfuseimage{godel}}


%%%% Delete this, if you do not want the table of contents to pop up at
%%%% the beginning of each subsection:
%%%\AtBeginSubsection[]
%%%{
%%%  \begin{frame}<beamer>
%%%    \frametitle{Outline}
%%%    \tableofcontents[currentsection,currentsubsection]
%%%  \end{frame}
%%%}

\begin{document}


%%%\begin{frame}
%%%  \frametitle{Outline}
%%%  \tableofcontents
%%%  % You might wish to add the option [pausesections]
%%%\end{frame}


% Since this a solution template for a generic talk, very little can
% be said about how it should be structured. However, the talk length
% of between 15min and 45min and the theme suggest that you stick to
% the following rules:  

% - Exactly two or three sections (other than the summary).
% - At *most* three subsections per section.
% - Talk about 30s to 2min per frame. So there should be between about
%   15 and 30 frames, all told.

\beamerdefaultoverlayspecification{<+->}
%%%\subsection[Short First Subsection Name]{First Subsection Name}

\begin{frame}
  \titlepage
\end{frame}

\begin{frame} 
  \frametitle{Test slide}

\begin{theorem}
This is an ordinary theorem.
\end{theorem}

\begin{theoremFermat}
This is Fermat's Theorem.
\end{theoremFermat}

\begin{reimann}
This is the Riemann Hypothesis.
\end{reimann}

\begin{ACExample}
This is almost a counterexample.
\end{ACExample}

\end{frame}


\begin{frame} 
  \frametitle{Models realizing many types}

\pause
Throughout these slides, $T$ will be a complete theory
in a countable language which has infinite models.

\bigskip
\pause
By the Compactness Theorem, any model
of $T$ has an elementary extension that realizes all types.

\bigskip
\pause
One expects such an extension to behave like a
``completion'' or ``compactification''
of the original model.

\bigskip
\pause
    {\bf Defn.} Call a model $\mathbb S$ of $T$
\emph{weakly saturated}
if it realizes all types in $S_n(T)$ for all $n$.

\end{frame}


\begin{frame} 
  \frametitle{Models realizing many types}


\pause

The definition of ``weakly saturated model''
seems dual to the definition
of atomic model, \pause so
in an ideal world, the following would be true:
\pause

\begin{enumerate}
\item Countable weakly saturated models of $T$ would exist. \pause
\item Any two would be isomorphic. \pause
\item Any countable model of $T$ would embed
  elementarily into the weakly saturated model. \pause
\item Two tuples in a weakly saturated model
  would have the same type iff they
  differed by an automorphism. \pause
\end{enumerate}

\medskip

But all of these statements are false. \pause

The first statement becomes true provided $|S_n(T)|<2^{\aleph_0}$
for all $n$. \pause
And then all statements become true with ``$\omega$-saturated''
in place of ``weakly saturated''.
\end{frame}

\begin{frame} 
  \frametitle{The Ehrenfeucht theory}

  \pause

{\bf Example.}
Let $T$ be the theory of dense linear order without endpoints
expanded by a strictly increasing $\omega$-chain of constants. \pause

\bigskip

\begin{enumerate}
\item Signature involves $<, c_0, c_1, \ldots$ only. \pause

\item Axioms for $T$ = \pause
\begin{enumerate}
\item[(i)] Axioms of dense linear orders without endpoints. \pause
\item[(ii)] $c_i < c_{i+1}$ for each $i$. \pause
\end{enumerate}

\item Theory has q.e. and is complete by modhw3p3 (Eblen, Murali, Ornstein). \pause

\item   ($I(T,\omega)=3$.) \pause
  Theory has three isomorphism types of countable models. \pause
Any countable model is isomorphic
to one of the form $\langle \mathbb Q; <, c_0, c_1,\ldots\rangle$
where \pause
\begin{enumerate}
\item[(i)] (Model $\mathbb M_1$)
  The sequence $(c_i)_{i\in\omega}$ is unbounded. \pause
\item[(ii)] (Model $\mathbb M_2$)  The sequence $(c_i)_{i\in\omega}$ has a least upper bound in the model. \pause
\item[(iii)] (Model $\mathbb M_3$)  The sequence $(c_i)_{i\in\omega}$ has an upper bound in the model but has no least upper bound in the model. \pause
\end{enumerate}
\end{enumerate}

\end{frame}

\begin{frame} 
  \frametitle{The countable models $\mathbb M_1, \mathbb M_2, \mathbb M_3$}
  \pause

\begin{figure}[!htbp]
\setlength{\unitlength}{1truemm}
\begin{picture}(100,60)
\put(0,0)
{
\begin{tikzpicture}[scale=.7]
\node at (1,5.5){Top};
\node at (1.5,2){Bottom};
\node at (1.5,1){$=\bigcup_i \; (c_i]$};

\draw [fill] (4,1.5) circle [radius=0.08];
\draw [fill] (4,2.1) circle [radius=0.08];
\draw [fill] (4,2.6) circle [radius=0.08];
\draw [fill] (4,3.0) circle [radius=0.08];
\draw [fill] (4,3.3) circle [radius=0.08];
\draw [fill] (4,3.5) circle [radius=0.08];
\node at (4.4,1.4){$c_0$};
\node at (4.4,2.0){$c_1$};
\node at (4.4,2.5){$c_2$};
\draw[densely dotted, line width=1pt] (4,0) -- (4,3.9);
%\draw[line width=1pt] (4,4.1) -- (4,7);
\node at (4.1,-.8){$\mathbb M_1$};
\draw[black, line width=1pt] (3.84,0.05) arc (230:310:.25);
\draw[black, line width=1pt] (4.16,3.85) arc (50:130:.25);

\draw [fill] (8,1.5) circle [radius=0.08];
\draw [fill] (8,2.1) circle [radius=0.08];
\draw [fill] (8,2.6) circle [radius=0.08];
\draw [fill] (8,3.0) circle [radius=0.08];
\draw [fill] (8,3.3) circle [radius=0.08];
\draw [fill] (8,3.5) circle [radius=0.08];
\node at (8.4,1.4){$c_0$};
\node at (8.4,2.0){$c_1$};
\node at (8.4,2.5){$c_2$};
\draw[densely dotted, line width=1pt] (8,0) -- (8,3.9);
\draw[densely dotted, line width=1pt] (8,4.2) -- (8,7);
\node at (8.1,-.8){$\mathbb M_2$};
\draw[black, line width=1pt] (7.84,0.05) arc (230:310:.25);
\draw[black, line width=1pt] (8.16,3.85) arc (50:130:.25);
%\draw[black, line width=1pt] (7.84,4.25) arc (230:310:.25);
\draw [fill] (8,4.25) circle [radius=0.08];
\draw[black, line width=1pt] (8.16,7) arc (50:130:.25);


\draw [fill] (12,1.5) circle [radius=0.08];
\draw [fill] (12,2.1) circle [radius=0.08];
\draw [fill] (12,2.6) circle [radius=0.08];
\draw [fill] (12,3.0) circle [radius=0.08];
\draw [fill] (12,3.3) circle [radius=0.08];
\draw [fill] (12,3.5) circle [radius=0.08];
\node at (12.4,1.4){$c_0$};
\node at (12.4,2.0){$c_1$};
\node at (12.4,2.5){$c_2$};
\draw[densely dotted, line width=1pt] (12,0) -- (12,3.9);
\draw[densely dotted, line width=1pt] (12,4.2) -- (12,7);
\node at (12.1,-.8){$\mathbb M_3$};
\draw[black, line width=1pt] (11.84,0.05) arc (230:310:.25);
\draw[black, line width=1pt] (12.16,3.85) arc (50:130:.25);

\draw[black, line width=1pt] (11.84,4.25) arc (230:310:.25);
\draw[black, line width=1pt] (12.16,7) arc (50:130:.25);

%\draw[line width=1.2pt] (1,0) -- (1,1);
%\draw[line width=1.2pt] (1,4) ellipse (1 and 3);

\draw[dashed, line width=1.2pt] (2,4) -- (15,4);
\end{tikzpicture}
}
\end{picture}
\begin{caption}
  {$\textrm{Top}(x)=\{c_0<x, \;\;c_1<x, \;\;c_2<x,\;\ldots\}$,
  nonisolated $p\in S_1(T)$}
%  By q.e., this is a basis for a complete type
\end{caption}
\end{figure} 
\end{frame}

\begin{frame} 
  \frametitle{Observations}
  \pause
\begin{enumerate}
\item The fact that $I(T,\omega)=3$ can be checked
  by noting that a countable model is determined
  up to isomorphism by the part above all the constants, \pause
  and that part is a (possibly empty) dense linear order
  without top element. \pause
\item All embeddings between models are elementary by q.e. \pause

  $\mathbb M_1\prec \mathbb M_2 \prec \mathbb M_3 \prec \mathbb M_2$. \pause
%\item $\mathbb M_1$ is atomic; \pause
%$\mathbb M_2$ is weakly saturated, but not $\omega$-saturated; \pause
%$\mathbb M_3$ is $\omega$-saturated.
\item $I(T,\omega)<2^{\aleph_0}$ implies $S_n(T)$ is scattered for all $n$, \pause
  so one of the models must
  be atomic. \pause The only plausible candidate is $\mathbb M_1$. \pause
\item All countable models embed elementarily into both
  $\mathbb M_2$ and $\mathbb M_3$. \pause
  This is enough to prove that $\mathbb M_2$ and $\mathbb M_3$
  are both weakly saturated. \pause
%\item The type of any tuple $\wec{m}\in \mathbb M_i^n$ is determined
%  by specifying whether $m_j\Box m_k$ or $m_j\Box c_k$ where
%  $\Box\in\{=, <, >\}$. \pause
\item The model $\mathbb M_2$ does
  not have the type-extension property. \pause

  Let $p\in S_1(T)$ be the type $p(x_1)=\textrm{Top}(x_1)$. \pause
  Let $q\in S_2(T)$ be the type
  $q(x_1,x_2)=\textrm{Top}(x_1)\cup\textrm{Top}(x_2)\cup\{x_2<x_1\}$. \pause
  $q|_1=p$. \pause
  Let $a=\textrm{lub}(c_i)$. \pause

  The $1$-tuple $(a)$ realizes $p$,
  \pause
  Some $1$-tuples that realize $p$ can be extended to $2$-tuples
  that realize $q$. \pause
  But the $1$-tuple $(a)$ cannot be extended to a $2$-tuple that realizes
  $q$.
\end{enumerate}
\end{frame}


\begin{frame} 
  \frametitle{Tweaking the example by coloring the points}
  \pause
  We introduce two new unary relations, ${\textrm{red}}(x)$
  and ${\textrm{blue}}(x)$. \pause
  Our goal is to construct a theory like Ehrenfeucht's,
  but with every point colored either red or blue, but not both
  colors. \pause

\begin{enumerate}
\item Signature involves
  $<, {\textrm{red}}(x), {\textrm{blue}}(x),  c_0, c_1, \ldots$ only. \pause

\item Axioms for $T$ = \pause
\begin{enumerate}
\item[(i)] Axioms of linear orders without endpoints. \pause
\item[(ii)] An axiom saying that each point has a unique color: \pause
  \[
\hspace{-.8in}{(\forall x)(
  ({\textrm{red}}(x)\wedge \neg {\textrm{blue}}(x)) \vee
  (\neg {\textrm{red}}(x)\wedge {\textrm{blue}}(x))
  )}.
  \]
\pause
\vspace{-.2in}
\item[(iii)] Both red points and blue points are dense: \pause
\[
\hspace{-.8in}{(\forall w)(\forall x)
((w<x)\to (\exists y)(\exists z)({\textrm{red}}(y)\wedge {\textrm{blue}}(z)
\wedge (w<y<x)\wedge (w<z<x)))}.
\]
\pause
\vspace{-.2in}
\item[(iv)] $c_i < c_{i+1}$ for each $i$. \pause
\item[(v)] ${\textrm{red}}(c_i)$ for each $i$. \pause
\end{enumerate}
\end{enumerate}
\end{frame}


\begin{frame} 
  \frametitle{This theory has four countable models}
  \pause

\begin{figure}[!htbp]
\setlength{\unitlength}{1truemm}
\begin{picture}(100,60)
\put(0,0)
{
\begin{tikzpicture}[scale=.7]
\node at (1,5.5){Top};
\node at (1.5,2){Bottom};


\draw[blue, dotted, line width=1pt] (4,0.08) -- (4,3.9);
\draw[red, dotted, line width=1pt] (4,0) -- (4,3.82);
%\draw[blue, dotted, line width=1pt] (4,4.28) -- (4,7);
%\draw[red, dotted, line width=1pt] (4,4.2) -- (4,6.92);

\draw [red, fill] (4,1.5) circle [radius=0.08];
\draw [red, fill] (4,2.1) circle [radius=0.08];
\draw [red, fill] (4,2.6) circle [radius=0.08];
\draw [red, fill] (4,3.0) circle [radius=0.08];
\draw [red, fill] (4,3.3) circle [radius=0.08];
\draw [red, fill] (4,3.5) circle [radius=0.08];
\node at (4.4,1.4){$c_0$};
\node at (4.4,2.0){$c_1$};
\node at (4.4,2.5){$c_2$};
\node at (4.1,-.8){$\mathbb N_1$};
\draw[black, line width=1pt] (3.84,0.05) arc (230:310:.25);
\draw[black, line width=1pt] (4.16,3.85) arc (50:130:.25);
%\draw[black, line width=1pt] (7.84,4.25) arc (230:310:.25);
%\draw [fill] (4,4.25) circle [radius=0.08];
%\draw[black, line width=1pt] (4.16,7) arc (50:130:.25);


\draw[blue, dotted, line width=1pt] (7,0.08) -- (7,3.9);
\draw[red, dotted, line width=1pt] (7,0) -- (7,3.82);
\draw[blue, dotted, line width=1pt] (7,4.28) -- (7,7);
\draw[red, dotted, line width=1pt] (7,4.2) -- (7,6.92);

\draw [red, fill] (7,1.5) circle [radius=0.08];
\draw [red, fill] (7,2.1) circle [radius=0.08];
\draw [red, fill] (7,2.6) circle [radius=0.08];
\draw [red, fill] (7,3.0) circle [radius=0.08];
\draw [red, fill] (7,3.3) circle [radius=0.08];
\draw [red, fill] (7,3.5) circle [radius=0.08];
\node at (7.4,1.4){$c_0$};
\node at (7.4,2.0){$c_1$};
\node at (7.4,2.5){$c_2$};
\node at (7.1,-.8){$\mathbb N_2$};
\draw[black, line width=1pt] (6.84,0.05) arc (230:310:.25);
\draw[black, line width=1pt] (7.16,3.85) arc (50:130:.25);
%\draw[black, line width=1pt] (7.84,4.25) arc (230:310:.25);
\draw [red, fill] (7,4.25) circle [radius=0.08];
\draw[black, line width=1pt] (7.16,7) arc (50:130:.25);

\draw[blue, dotted, line width=1pt] (10,0.08) -- (10,3.9);
\draw[red, dotted, line width=1pt] (10,0) -- (10,3.82);
\draw[blue, dotted, line width=1pt] (10,4.28) -- (10,7);
\draw[red, dotted, line width=1pt] (10,4.2) -- (10,6.92);

\draw [red, fill] (10,1.5) circle [radius=0.08];
\draw [red, fill] (10,2.1) circle [radius=0.08];
\draw [red, fill] (10,2.6) circle [radius=0.08];
\draw [red, fill] (10,3.0) circle [radius=0.08];
\draw [red, fill] (10,3.3) circle [radius=0.08];
\draw [red, fill] (10,3.5) circle [radius=0.08];
\node at (10.4,1.4){$c_0$};
\node at (10.4,2.0){$c_1$};
\node at (10.4,2.5){$c_2$};
\node at (10.1,-.8){$\mathbb N_3$};
\draw[black, line width=1pt] (9.84,0.05) arc (230:310:.25);
\draw[black, line width=1pt] (10.16,3.85) arc (50:130:.25);
%\draw[black, line width=1pt] (7.84,4.25) arc (230:310:.25);
\draw [blue, fill] (10,4.25) circle [radius=0.08];
\draw[black, line width=1pt] (10.16,7) arc (50:130:.25);

\draw[blue, dotted, line width=1pt] (13,0.08) -- (13,3.9);
\draw[red, dotted, line width=1pt] (13,0) -- (13,3.82);
\draw[blue, dotted, line width=1pt] (13,4.28) -- (13,7);
\draw[red, dotted, line width=1pt] (13,4.2) -- (13,6.92);

\draw [red, fill] (13,1.5) circle [radius=0.08];
\draw [red, fill] (13,2.1) circle [radius=0.08];
\draw [red, fill] (13,2.6) circle [radius=0.08];
\draw [red, fill] (13,3.0) circle [radius=0.08];
\draw [red, fill] (13,3.3) circle [radius=0.08];
\draw [red, fill] (13,3.5) circle [radius=0.08];
\node at (13.4,1.4){$c_0$};
\node at (13.4,2.0){$c_1$};
\node at (13.4,2.5){$c_2$};
\node at (13.1,-.8){$\mathbb N_4$};
\draw[black, line width=1pt] (12.84,0.05) arc (230:310:.25);
\draw[black, line width=1pt] (13.16,3.85) arc (50:130:.25);
\draw[black, line width=1pt] (12.84,4.25) arc (230:310:.25);
%\draw [fill] (13,4.25) circle [radius=0.08];
\draw[black, line width=1pt] (13.16,7) arc (50:130:.25);



\draw[dashed, line width=1.2pt] (2,4) -- (15,4);

\end{tikzpicture}
}
\end{picture}
\begin{caption}
  {\sc $\mathbb N_1\prec\mathbb N_2\prec\mathbb N_3\prec\mathbb N_4\prec\mathbb N_2$}
\end{caption}
\end{figure}
\end{frame}



\begin{frame} 
  \frametitle{More observations about the uncolored version}

\pause

\begin{enumerate}
\item $\mathbb M_2$ \pause
  (the model where $\textrm{lub}(c_i)$ exists in the model) \pause
  does not have the type-extension
  property. \pause
The problem involves the bound $a = \textrm{lub}(c_i)$. \pause
\item If $a=\textrm{lub}(c_i)$,
  then $p=\{c_i<x<a\;|\;i\in\omega\}$ is a
  $1$-type in $L(\{a\})$, which is not realized
  in $(\mathbb M_2)_a$. \pause
  Thus, $\mathbb M_2$ is weakly saturated, while
  an expansion by a single constant is no longer
  weakly saturated. \pause
\item All upper bounds of the sequence $(c_i)_{i\in\omega}$
  have the same $1$-type over the empty set \pause
  (namely $\textrm{Top}(x)$). \pause
  But $a = \textrm{lub}(c_i)$ does not differ from
  other realizations of $\textrm{Top}(x)$ by an automorphism. \pause
\item On the other hand, $\mathbb M_3$ does have the type-extension
  property, \pause any expansion of $\mathbb M_3$ by finitely
  many constants is again weakly saturated, \pause
  and any two tuples of the same type
  in $\mathbb M_3$ differ by an automorphism. \pause
  $\mathbb M_3$ is $\omega$-saturated.
\end{enumerate}

\end{frame}


\begin{frame} 
  \frametitle{Isomorphism}

  \pause
  Let $\mathbb A$ and $\mathbb B$ be countable structures
  both enumerated by $\omega$: \pause

$\mathbb A = \{a_0, a_1, a_2, \ldots\}$ \pause
 
$\mathbb B = \{b_0, b_1, b_2, \ldots\}$ \pause

  The assignment $a_i\mapsto b_i$ is an isomorphism iff it is type-preserving:
  \pause

  \begin{equation}\label{type_eq}
\textrm{tp}(a_0\cdots a_{n-1}) = \textrm{tp}(b_0\cdots b_{n-1}) 
  \end{equation}
  for all $n$.  \pause

  \bigskip

Suppose we want to build an isomorphism one element at a time,
by ensuring that, \pause
given equality of types of length-$n$
  initial segments $\wec{a}, \wec{b}$, as in (\ref{type_eq}),
  and given the choice for $a_n$, we can find
  a corresponding choice for $b_n$. \pause
  If we work only with types over the empty set,
  then we need some form of 
  the type-extension lemma. \pause
  It is enough to assume $\mathbb A$ and $\mathbb B$ are weakly
  saturated \pause
  PLUS any two tuples of the same type differ by an automorphism.
  \pause
  OR, we can work with $\mathbb A_{\wec{a}}$
  and $\mathbb B_{\wec{b}}$ and then deal only with types
  in the expanded language $L(\wec{a})$. \pause
  
\end{frame}



\begin{frame} 
  \frametitle{$\omega$-saturation}

\pause
\noindent
{\bf Defn.} Let $T$ be a complete theory. \pause
\begin{enumerate}
\item A model $\mathbb M$ of $T$ is \emph{$\omega$-saturated}
  if, \pause whenever $\wec{a}\in M^n$, $\mathbb M_{\wec{a}}$
  realizes every type in $S_1(\wec{a})$.
  \pause
  Often written ``whenever $A\subseteq \mathbb M$, $|A|<\omega$,
  $\mathbb M_A$ realizes every type
  in $S_1(A)$''. \footnote{Equivalently,
$\mathbb M_{A}$
    realizes every type in $S_n(A)$ for each finite $n$,
  Proposition 4.3.2, Marker.}
  \pause
\item (Type extension) \pause
  A model $\mathbb M$ of $T$ is \emph{$\omega$-homogeneous}
  if, \pause
  whenever $\wec{a}, \wec{b}\in M^n$, 
  $\textrm{tp}(\wec{a})=\textrm{tp}(\wec{b})$,
  and $c\in M$, then there exists $d\in M$
  such that   $\textrm{tp}(\wec{a}c)=\textrm{tp}(\wec{b}d)$. \pause
\item 
  A model $\mathbb M$ of $T$ is \emph{strongly $\omega$-homogeneous}
  if, \pause
  whenever $\wec{a}, \wec{b}\in M^n$, 
  $\textrm{tp}(\wec{a})=\textrm{tp}(\wec{b})$,
  then there is an automorphism $\alpha$ of $\mathbb M$
  such that $\alpha(\wec{a})=\wec{b}$. \pause
\item A model $\mathbb M$ of $T$ is \emph{$\omega^+$-universal}
  every countable model of $T$ is elementarily
  embeddable in $\mathbb M$. \pause
  (In particular, an $\omega^+$-universal model will be weakly saturated.)
\end{enumerate}
        
\end{frame}


\begin{frame} 
  \frametitle{Relationships}

\pause
\noindent
    {\bf Theorem.} Let $T$ be a complete theory in a countable language.
    TFAE about a countable model $\mathbb M$ of $T$. \pause
\begin{enumerate}
\item $\mathbb M$ is $\omega$-saturated. \pause
\item $\mathbb M$ is weakly saturated and $\omega$-homogeneous.
  ($\mathbb M$ realizes all types over the empty set
  and has the type extension property.) \pause
\item $\mathbb M$ is weakly saturated and strongly
  $\omega$-homogeneous. \pause
\item $\mathbb M$ is $\omega^+$-universal and $\omega$-homogeneous. \pause
\item $\mathbb M$ is $\omega^+$-universal and
  strongly $\omega$-homogeneous. \pause
\end{enumerate}

\medskip
\pause
\noindent
{\bf Trivial implications.} \pause

$\omega^+$-universality implies weak saturation.  \pause

Strong $\omega$-homogeneity implies $\omega$-homogeneity. \pause

\medskip
\pause
\noindent
{\bf Not-too-hard implications.} \pause

$\omega$-saturation implies strong $\omega$-homogeneity. \pause
(Back and forth.) \pause

$\omega$-saturation implies $\omega^+$-universality.  \pause
(Forth.) \pause

\end{frame}


\begin{frame} 
  \frametitle{A proof sketch}

\pause
\noindent
    {\bf Theorem.}
    Two countable $\omega$-saturated models of $T$ are isomorphic. \pause
(Back and forth.) \pause

\bigskip

Assume $\mathbb A$ and $\mathbb B$ are $\omega$-saturated models of $T$. \pause

\medskip

Enumerate them. \pause

\medskip

Start back and forth: \pause Assume that $f: \wec{a}\to \wec{b}$
is a partial isomorphism that we want to extend. \pause
At this point,
$\textrm{tp}^{\mathbb A}(\wec{a})=\textrm{tp}^{\mathbb B}(\wec{b})$. \pause
Equivalently, $\mathbb A\models \varphi(\wec{a})$
iff $\mathbb B\models \varphi(\wec{b})$. \pause
Equivalently, $\mathbb A_{\wec{a}}\equiv \mathbb B_{\wec{b}}$. \pause

\medskip

Assume it is our turn to extend the domain. \pause
Let $c\in \mathbb A$ be the least unconsidered element. \pause
Let $p=\textrm{tp}^{\mathbb A_{\wec{a}}}(c)$. \pause
Let $d$ be a realization of $p$ in $\mathbb B_{\wec{b}}$. \pause
Thus, $\mathbb A\models \theta(\wec{a}c)$
iff $\mathbb B\models \theta(\wec{b}d)$. \pause
I.e.,  $\textrm{tp}^{\mathbb A}(\wec{a}c)=\textrm{tp}^{\mathbb B}(\wec{b}d)$. \pause
Extend $f$ so that $f(c)=d$. \pause $\Box$ \pause
\bigskip

When $\mathbb A=\mathbb B$, this argument proves strong $\omega$-homogeneity
of $\omega$-saturated models. \pause
Half of the argument proves $\omega$-universality.
\end{frame}


\begin{frame} 
  \frametitle{Existence and uniqueness}

\bigskip
\pause

\noindent
    {\bf Theorem.} Let $T$ be a complete theory in a countable language.
    If $\mathbb A$ and $\mathbb B$ are countable $\omega$-saturated models
    of $T$, then $\mathbb A\cong \mathbb B$.     

\pause
\noindent
    {\bf Theorem.} Let $T$ be a complete theory in a countable language.
TFAE. \pause
\begin{enumerate}
\item $T$ has a countable weakly saturated model.
\item $T$ is ``small''. ($|S_n(T)|<2^{\aleph_0}$ for all $n$.)
\item $T$ has a countable $\omega$-saturated model.
\end{enumerate}
\end{frame}

\end{document}

